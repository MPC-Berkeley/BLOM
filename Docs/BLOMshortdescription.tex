\documentclass{article}
\usepackage{fullpage,amsmath}
\begin{document}
\section{BLOM}
\subsection{Installation}
Follow the instructions below to download and install BLOM.
\begin{enumerate}
\item Go to the BLOM download page: http://www.mpclab.net/Trac/wiki/BLOMSetup
\item Download BLOM ``As a zip file"
\item Extract the ``trunk" folder into your ``H:" drive
\item Start Matlab and type \texttt{addpath(genpath('H:$\backslash$trunk'))} into the command window. This will add BLOM to your path.
\item Type \texttt{RunHelloWorldR2} into the command window to run the test program.
\end{enumerate}
\subsection{Polyblocks}
BLOM supports functions that are a quotient of polynomial-like functions with a special structure. These polynomial-like functions must be a sum of terms which are products of powers or ``basic'' functions of the variables. The basic functions currently supported are exponentials, logarithms, sines, cosines, hyperbolic tangents, and inverse tangents. More precisely, the functions must be of the form
\[h(x_1,...,x_n)=\frac{f(x_1,...,x_n)}{g(x_1,...,x_n)},\text{ where }f(x_1,...,x_n)=\sum_{i=1}^p \prod_{j=1}^n a_{ij}(x_j),\:g(x_1,...,x_n)=\sum_{i=1}^q \prod_{j=1}^n b_{ij}(x_j),\]
and $a_{ij}$ and $b_{ij}$ are powers or basic functions.
For example, the following function is acceptable in BLOM.
\[h(x_1,x_2,x_3)=\frac{f(x_1,x_2,x_3)}{g(x_1,x_2,x_3)}=\frac{\overbrace{x_1^2x_2\cos(x_3)}^{\text{term 1}}+\overbrace{e^{x_1}x_2^3x_3^5}^{\text{term 2}}+\overbrace{x_1^2\log (x_2)x_3^3}^{\text{term 3}}}{\underbrace{x_1\tan^{-1}(x_2)}_{\text{term 1}}+\underbrace{\tanh(x_2)x_3^5}_\text{term 2}}\]
In Simulink, the user uses the `PolyBlock' function block to program these functions. Multiple functions can be specified at the same time using one PolyBlock. The numerator, $f(x)$, and denominator, $g(x)$, are specified separately using a terms matrix $A$ and coefficient matrix $C$ for each.

The $ij$-th entry of $A$ corresponds to the power or basic function of the $j$-th variable in the $i$-th term. Therefore, the $i$-th row of $A$ corresponds to the $i$-th term in the polynomial-like function. The $ij$-th entry of $C$ corresponds to the coefficient of the $j$-th term in the $i$-th polynomial-like function. Therefore, the $i$-th row of $C$ represents the $i$-th polynomial-like function.

For example, suppose we would like to program the following two functions using one PolyBlock.
\begin{gather*}
h_1(x_1,x_2)=\frac{f_1(x_1,x_2)}{g_1(x_1,x_2)}=\frac{x_1^2x_2+2x_1\cos(x_2)}{x_2}\\
h_2(x_1,x_2)=\frac{f_2(x_1,x_2)}{g_2(x_1,x_2)}=\frac{3\sin(x_1)+e^{x_2}}{\tanh(x_1)x_2^3+4x_2}
\end{gather*}
For the numerators $f_1(x)$ and $f_2(x)$, the $A$ and $C$ matrices are
\[A=\begin{bmatrix}2&1\\1&\texttt{BLOM\_FunctionCode('cos')}\\\texttt{BLOM\_FunctionCode('sin')}&0\\0&\texttt{BLOM\_FunctionCode('exp')}\end{bmatrix},\: C=\begin{bmatrix}1&2&0&0\\0&0&3&1\end{bmatrix}\]
For the denominators $g_1(x)$ and $g_2(x)$, the $A$ and $C$ matrices are
\[A=\begin{bmatrix}0&1\\\texttt{BLOM\_FunctionCode('tanh')}&3\end{bmatrix},\: C=\begin{bmatrix}1&0\\4&1\end{bmatrix}\]
\end{document}

