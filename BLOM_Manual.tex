
\documentclass {article}
 
\usepackage{graphicx}
\usepackage{graphics}
\usepackage[left=2cm,top=1cm,right=2cm,nohead,nofoot]{geometry}
\usepackage{setspace}
\usepackage{amsmath}
	

\usepackage{amssymb}


\doublespacing

\usepackage{amsfonts}
  \newcommand{\field}[1]{\mathbb{#1}}
  \newcommand{\reals}{\field{R}}


\begin{document}


\title{ Manual: using optimization modeling library. \\
Berkeley Library for Optimization Modeling:  BLOM\\
Revision 0.1 }
\author{Sergey Vichik}

\maketitle


\section{Introduction}

The library and the associated tools are designed to ease process of modeling
dynamical nonlinear systems for optimization problems, especially MPC. 
The library allows intuitive block diagram definition of the system,
simulation, validation, and an automatic generation of an optimization problem
to be solved. This process eliminates errors introduced by manual conversion
and allows conversion of large models, too large to be converted manually. 
The library consists of Simulink blocks library and Matlab software for
converting model to an optimization problem. 

This library supports the following work flow for defining an optimization problems. 
First the model of the system is drawn using Simulink and the supplied blocks. 
The models includes 
\begin{enumerate}
\item 
All equations that describe the system
\item Inequality constraints 
\item Cost function
\item Which variable is the free variable to optimize, and what variable is externally determined.
\end{enumerate}
In the second stage the model is validated by running forward mode
simulation. The validation includes a check whether constraints are violated. 
After the validation phase, the model is converted to optimization problem, thus
\begin{enumerate}
\item Cost function.
\item Linear and nonlinear equality constraints (from model equations)
\item Linear and nonlinear inequality constraints.
\item List of optimization variables.
\item Externally specified variables using user supplied data.
\end{enumerate}
If the model includes dynamics, like MPC problems, the optimization problem
includes the requested number of time steps. 

Finally, the optimization problem is exported to one of the following
optimization engines, including automatic Matlab or C code generation  
\begin{enumerate}
\item Matalb's fmincont
\item TOMLAB - TBD
\item CPLEX - TBD
\item IPOPT - TBD
\end{enumerate}



\section{Modeling blocks} 
For model definition only the blocks listed in the library are allowed. All
other Simulink blocks can be used outside of the modeled subsystem. 

\subsection{PolyBlock}
PolyBlock  is a key element of the library.
\begin{description}
  \item [Responsibility:] - Defines output as function of input. The function
    is defined using ....
\item [Input:] input variables, as many as width of A matrix.
\item [Output] the calculated value.
\end{description}

\subsubsection{Remarks}
 If the expression is short, it will be shown on the block. If the expression is too long, only y=f(x)/g(x) will be shown.

More versions of the block exist in the library, for convenience only:
\begin{description}
  \item [Sum operator] sum of two variables,
\item [Difference operator] - difference of two variables,
\item [Product] - product of two variables.
\item [Constant gain]
\item [Bias]
\item [General form PolyBlock] - can be used to define any relation between input and output, including non-functions, like solution of quadratic equation. In this case, the behavior in forward simulation is undefined. 
\end{description}

\subsubsection{Usage}

Basic building block is a sum of terms of the following form $\sum_i \prod_j
v_{i,j}(x_i)$, where $v(x_i)$ can be $x^p$ for any $p \in R$, $\exp(x)$ or
$\log(x)$. 
Functions are defined using two matrices, called here $A$ and $C$.  Where
$A_{i,j} = p$ where $p$ is a power of $v_{i,j}$ function.  
Two special values are reserved for $p$ to mark the $\exp$ and the $\log$
functions. $p = \infty $ marks $\exp$ and $p =-\infty$ marks $\log$.
Although $p$ can be any real number, only positive integers and $\exp$ must be
used for most cases, because their provide the fastest computation time. 
Library supports other options for completeness and to accommodate very
special cases that cannot be handled with the fast functions. 
Example:
TBD
  
The PolyBlock enables definition of rational function of the form $\frac{f(x)}{g(x)}$ where $f$ and $g$ are functions as defined above.

Following special cases of PolyBlock are important:
\begin{description}
  \item [Constant] $A = [0 \ldots 0]$ , output $= C$.
\item [Linear algebra] If $A=I$, then the output of the block is $Cx$ where
  $x$ is the input and $C$ is any matrix of compatible width. 
\item [Vector functions] Vector functions can be defined. The number of rows
  of C matrix defines the dimension of output vector.
 
\item [Selector function] each row of A holds $1$ at requested position, and
  $C=I$. 
\end{description}

The dimensions of the A and C matrices of $f$ and $g$ functions, must follow
the following rules. If:\\
\begin{tabular}{ll}
$n$ &input dimension
 \\
$m$ & output dimension
 \\
$r$ & number of term in f function \\
$k$ & number of terms in g function
 \\
\end{tabular} \\
Then: \\
\begin{align*}
fA &\in \reals^{r x n} \\
fC &\in \reals^{m x r}
 \\
gA &\in \reals^{k x n}
 \\
gC &\in \reals^{1 x k}
 \\
\end{align*}
note that $g$ must be scalar function, because division by vector is not defined.

\subsection{ Constraint}
\begin{description}
  \item [Responsibility] - Marks variable as a constrained to be $ >0$ or $<0$. Turns red if the constraint is violated during the simulation.
\item [Input] The variable to be constrained.
\item [Output] none.
\end{description}

\subsection{ Input}
\begin{description}
  \item [Responsibility] - Marks a variable as an input variable. Can specify the number of steps to freeze the input.
 \item [Input]  a variable
\item [Output] - equal to input
\end{description}

\subsection{ External}
\begin{description}
  \item [Responsibility] - Marks a variable as an external variable. 
 \item [Input]  a variable
\item [Output] - equal to input
\end{description}

\subsection{ State}
\begin{description}
  \item [Responsibility] - Holds value for the next step.
  \item [Input]  a variable
  \item [Output]  value from the previous step.
\end{description}

\subsection{ Cost}
\begin{description}
  \item [Responsibility]  defines cost function. Sums input variable on every
    time step to calculate the cost.
  \item [Input]  variable
  \item [Output] none
\end{description}

\subsection{ Save to workspace}
\begin{description}
  \item [Responsibility]  copy of the regular Simulink block. Required for storing variables for initial optimization solution.
  \item [Output] none
\end{description}



\section{Generation of an optimization problem }

TBD

\end{document}
